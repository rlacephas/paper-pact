
\def\graphwidth{0.9\linewidth}


\REM{ % irrlelvent!
running at 2000Mhz, with 128KB L1, 512KB L2, 2MB L3 shared, and 4 GB
of memory.
}

The first experiment was to tune in the value of the compiler parameters.
 A machine learning algorithm was used to find the sweetspot for a set of
 inlining parameters. As described in Section~\ref{sec:ml}, we used
 SPSA - Simultaneous Perturbation Stochastic Approximation because we have no
 supposition on the function and on the search space.

The second experiment was performed to evaluate and compare two different cases,
 the case of single-runs, and the case of \CP\-runs. Both cases are analyzed and
 compared.

The third experiment was conducted to verify and 
  analyze behavior variability for random choice, simple sort and
  a version of the knapsack problem. Particularly the latter version presented better results.

\subsection{Programs and Inputs}


This study evaluates the inliners described above using four
programs: \bzip, \gzip, \gcc, and \gobmk.  Each program is evaluated
using a 15-input workload, as suggested
in \cite{BerubePhD}.  \Gcc\ and \gobmk\ are taken from the
SPEC CPU 2006 benchmark suite.  SPEC provides 11 inputs for \gcc.  In
spite of the challenges involved in creating new inputs for this
benchmark, four\footnote{of seven attempts} of the SPEC 2000 benchmark
programs were converted to the single pre-processed file format.  The
converted programs are \bzip, \lbm, \mcf, and \parser.  For \gobmk,
SPEC provides 20 inputs.  However, only 5 of these inputs come from
the {\tt ref} workload; the {\tt train} workload contains 8 inputs,
and the {\tt test} workload contains 7 inputs.  Many of the inputs from
{\tt test} and {\tt train} have very short execution times: 4 inputs
take less than 1 second, 6 take 2--9 seconds, 4 take 12--19 seconds,
and 1 takes longer than 1 minute.  Execution times of less than a few
seconds are subject to large proportional timing imprecision, because
the Linux {\tt time} command reports times with a resolution of
1/100$^{th}$ of a second.  Therefore, the 15 longest-running inputs
are chosen for \Wfull.  This set is composed of the {\tt ref} and {\tt
train} SPEC workloads, plus \iname{connect} and \iname{dniwog} from
{\tt test}.  The shortest baseline running time in \Wfull\ is 2.3
seconds, for \iname{connect}.

The other two programs used in the case study are \bzip\ and \gzip.
However, rather than using the SPEC benchmark versions of these
programs, the fully-functional ``real'' versions are used.  Using the
real versions of the compressor programs eliminates the
unrealistically-simplified profiling situation where
mutually-exclusive use cases are combined into a single program run.
Consequently, these programs cannot do decompression and compression,
or multiple levels of compression, within the same run.  These
distinct use-cases must be covered by different inputs in the program
workload.  Both \bzip\ and \gzip\ share the same workload of inputs.
This workload is split in half into a compression set and a
decompression set.  Several inputs in the compression set have an
analogue in the decompression set.  However, the file format is
usually different, and the source of the data is never the same.  For
instance, \iname{revelation\mbox{-}ogg} in the compression set
and \iname{sherlock\mbox{-}mp3} in the decompression set are both audio
books, but the audio is recorded in different formats, and the books
themselves are different.

Both compressors use a numeric command-line flag to control the
tradeoff between compression speed and compression quality.  The flags
take integer values between 1 (fastest, least compressed) and 9
(slowest, most compressed).  The seven inputs in the compression set
each use a different compression level, from 3 to 9.  Most inputs are
collections of files.  Each collection is archived (uncompressed) so
that the input and output of each run is a single file.  In order to
minimize the impact of disk access, the output of each run is
redirected to {\tt /dev/null}.

The compression set contains the following inputs, with the
compression level shown in parentheses:
\begin{itemize}

\item {\tt avernum (-3)}: The installer for the demo version of the game
  ``Avernum: Escape from the Pit'' from Spiderweb Software.

\item {\tt cards (-4)}: A collection of greeting card layouts in the TIFF 
  (uncompressed) image format.

\item {\tt ebooks (-5)}: A collection of ebooks, with and without images,
  and in a variety of formats, from Project
  Gutenberg\footnote{http://www.gutenberg.org}.

\item {\tt potemkin-mp4 (-6)}: The 1925 movie ``Bronenosets Potyomkin
  (Battleship Potemkin)'' in MP4 format, from the Internet
  Archive\footnote{http://archive.org/details/BattleshipPotemkin}.

\item {\tt proteins-1 (-7)}: A sample of 33 proteins from the RCSB Protein Data
  Bank database.  6 files for each protein, each stored in a different
  text-based format, provide different characteristics of the protein's
  structure\footnote{http://www.rcsb.org}.

\item {\tt revelation-ogg (-8)}: The audio book ``The Revelation of Saint
  John'' in OGG format, from Project
  Gutenberg\footnote{http://www.gutenberg.org/ebooks/22945}.

\item {\tt usrlib-so (-9)}: A collection of shared object (.so) files from {\tt
  /usr/lib/} of a 32-bit gentoo-linux machine.

\end{itemize}

The decompression set for each compressor uses the same base set of
files, pre-compressed by the appropriate compressor at the default
compression level.  The decompression set is composed of:
\begin{itemize}
\item {\tt auriel}: The ``Auriel's Retreat'' land-mass addition mod by
  lance4791 for the game ``The Elder Scrolls IV: Oblivion'' from
  Bethesda
  Softworks\footnote{http://planetelderscrolls.gamespy.com/View.php?view=OblivionMods.Detail\&id=5949}.

\item {\tt gcc-453}: The source-code archive of the \gcc\ compiler,
  version 4.5.3\footnote{http://gcc.gnu.org/gcc-4.5}.

\item {\tt lib-a}: A collection of library files (.a) from {\tt /lib/} of a
  gentoo-linux machine.  As per the gentoo development guide, a
  library will be installed in {\tt /lib} (boot critical) or {\tt
    /usr/lib} (general applications), but not both\footnote{
    http://devmanual.gentoo.org/general-concepts/filesystem/index.html}.

\item {\tt mohicans-ogv}: The 1920 movie ``Last of the Mohicans'' in OGV
  (ogg video) format, from the Internet
  Archive\footnote{http://archive.org/details/last\_of\_the\_mohicans\_1920}.

\item {\tt ocal-019}: The Open Clip Art Library archive, version 0.19.  The
  images are primarily in vector-graphics
  formats\footnote{http://openclipart.org/collections}.

\item {\tt paintings-jpg}: A collection of watercolor paintings, in JPG format.

\item {\tt proteins-2}: A completely different sample of 157 proteins from
  the RCSB Protein Data Bank database, each in 6 different file
  formats.

\item {\tt sherlock-mp3}: The audio book ``The Adventures of Sherlock
  Holmes'' in MP3 format, from Project
  Gutenberg\footnote{http://www.gutenberg.org/ebooks/28733}.

\end{itemize}
