
The feedback-directed inlining (\FDI) ~\cite{BerubePhD} evaluated in this work is
fundamentally different than the existing static inliner in \llvm.
The static inliner inlines small calls to remove call overhead with
minimal increases in code size.  \FDI\ attempts to minimize the
dynamic number of instructions executed by the program by inlining the
most frequent calls.  While the static inliner considers call sites on
a function-by-function basis, \FDI\ considers the set of inlining
opportunities present at global scope in the current state of the
program.

% Remove the ``then'' keyword from if statements
\SetKwIF{If}{ElseIf}{Else}{if}{}{else if}{else}{endif}
\begin{algorithm}[t!p]
\begin{small}
  \SetKwInOut{Input}{input}\SetKwInOut{Output}{output}

  \Input{Module M: Whole-program IR}
  \Input{File cpFile: Combined profile}

  \KwData{List$<$call site$>$ candidates, ignored}
  \KwData{Map$<$Function $\rightarrow$ List$<$call site$>$ $>$ callers}

  initialize(M, cpFile)\label{wl:init}\;
  budget = computeCodeGrowthBudget()\label{wl:budget}\;
  candidates.sort()\;

  %\tcp{Inline the best candidate until the budget is exhausted or 
  %     there's nothing left to inline}
  \While{budget $> 0$  AND NOT candidates.empty}
  {
    %\tcp{remove best candidate}
    source = candidates.popBest()\label{wl:pop}\;

    %\tcp*{the best candidate should not be inlined!}
    \If{source.score $\le 0$\label{wl:ifscore}}
    {
      {\bf break}\;
    }

    %\tcp{already know that callee cannot be inlined}
    \If{source.callee.cannotInline\label{wl:ifcannot}}
    {
      ignored.add(source)\;
      {\bf continue}\;
    }

    %\tcp{too big for remaining budget; skip it}
    \If{source.expectedCodeGrowth $>$ budget\label{wl:ifbig}}
    {
      ignored.add(source)\;
      {\bf continue}\;
    }
      
    \tcp{Try to inline the candidate...}
    inlineResult = LLVM.inlineIfPossible(source)\label{wl:inline}\;
    \If{inlineResult.failed}
    {
      source.callee.setCannotInline()\;
      ignored.add(source)\;
      {\bf continue}\;
    }
%    {
      \tcp{Inlining succeeded}
      %\tcp{update budget using real code growht}
      budget -= inlineResults.codegrowth\label{wl:cost}\;
      %\tcp{the call site no longer exists}
      callers[source.getCallee].delete(source)\label{wl:delcaller}\;
      %\tcp{recalculate the score for the callers; their callee's size has changed}
      \For{caller $\in$ callers[source.caller]\label{wl:evalcallers}}
      {
        caller.calcScore()\;
      }
      %\tcp{deal with any new call sites created by inlining}
      \For{i $\leftarrow$ 1 \KwTo\ inlineResults.numInlinedCalls\label{wl:newcalls}}
      {
        target = inlineResults.inlinedCall[i]\label{wl:newcand}\;
        original = inlineResults.originalCall[i]\label{wl:orig}\;
        callers[target.getCallee].insert(target)\;

        \eIf{ignored.contains(original) $> 0$\label{wl:invalid}}
        {
          target.histogram = 0\;
          ignored.add(target)\;
        }
        {
          target.histogram = source.histogram $\times$ original.histogram\label{wl:newhist}\;
          target.calcScore()\;
          candidates.insert(target)\;
        } % eIf
      } % For
%    } % eIf
 }

\end{small}
  \caption{\FDI\ worklist}
  \label{alg:fdiworklist}
\end{algorithm}

\subsection{Worklist Algorithm}

\refAlgorithm{alg:fdiworklist} presents an outline of the worklist
algorithm used by \FDI.  The algorithm uses several data structures:
\begin{description}

\item[{\it candidates}]: \\
  The worklist is a sorted list of candidates.
  A call site is an inlining candidate if it is a direct call, and if
  the callee does not contain a \name{setjump} nor has any previous
  attempt to inline the callee failed.  Furthermore, the call site
  must have executed at least once during profiling.

\item[{\it ignored}]: \\
  A list of call sites that are not inlining
  candidates.  This list is maintained to enable correct and efficient
  bookkeeping, and to allow any copies of these call sites created by
  inlining their caller to be immediately ignored.

\item[{\it callers}]: \\
  A mapping from functions to the call sites that
  call them.  This map allows for the re-scoring of call sites on the
  event that a call is inlined into their callee.  That inlining will
  change the callee's size, and may change the expected
  simplifications possible if the callee is inlined.

\item[{\it inlineResult}]: \\
  A structure returned by inliner that
  provides summary information regarding the transformation.  In
  particular, it indicates if the attempted inlining failed.
  \FDI\ enhances the default \llvm\ structure with co-indexed lists
  identifying the new call sites created in the caller by inlining,
  and their originating call sites in the callee.  This information is
  required so that profile information can be estimated for the new
  call sites.

\end{description}

At the start of \FDI, the \CProf\ is read in, and the histograms are
associated with the appropriate call sites (\refLine{wl:init}).  Every
call site is inserted into the callers list of their callee.  During
initialization, each call site is evaluated, and added to either the
candidates or ignored list, as appropriate.  When a call site is
rejected for inlining, it is immediately and permanently moved from
the list of candidates to the ignore list.  Transformations such as
constant propagation or alias analysis can resolve the callee of an
indirect call to a single possibility, thereby making it a direct
call.  However, if the call is indirect when the call site is first
discovered by the inliner, it is placed on the ignore list in spite of
the possibility of future inlining resolving the call.  Calls to
libraries and compiler built-in functions are also immediately ignored
because they cannot be inlined.

\subsection{Candidate Scoring}
\label{inlining:scoring}

The \llvm\ inliner makes inlining decisions at each call site by
comparing the expected code growth to a fixed threshold.  \FDI\ takes
a more directly execution-time-oriented approach to inlining and
attempts to achieve the greatest reduction in executed instructions
for the least amount of code growth.  Therefore, \FDI\ breaks the
evaluation of a call site into three components: the expected inlining
benefit, the expected code growth, and execution frequency of the call
site.  Given a call site, \CS, the inlining candidate scoring
function, \ScoreOf{\CS}, combines these three elements so
that \refAlgorithm{alg:fdiworklist} can select the best (highest
score) candidates for inlining first. \CP\ provides a rich
characterization of execution frequency. Making use of that
information is described in detail in \cite{BerubePhD}; for
now, let $\RewardOf{\BenefitOf{\CS}, R_{\CS}}$ represent some function
of the estimated (execution-frequency independent) inlining benefit at
call site \CS\ and $R_{\CS}$, that call-site's \CP\ monitor.  Given
the benefit function \BenefitOf{\CS} and a cost function \CostOf{\CS}
described in this section, an inlining candidate's \Score\ is
conceptually computed:

$$ \ScoreOf{\CS} = \dfrac{\RewardOf{\BenefitOf{\CS}, R_{\CS}}}{\CostOf{\CS}} $$

More details on Benefit and Cost functions can be found in \cite{BerubePhD}.

\subsection{Finding the Best Candidate}
\label{inlining:candidate}

The best candidate to inlining is found by its own score. The way it is
done is simply sorting the candidates list by their score values. So
every time a new candidate is needed, just take the top one from the
ordered list of candidates.

But the ordering depends on scoring, and scoring is sensitive to some
parameters values. For instance, the expected reduction/expansion for each
function is directly dependent of its expected size, and the reduction/
expansion define the benefit and the cost of a function, in other words, its
own score. The parameters that can have some effect on the scoring function
are the sizes of a instruction, a call, a return, a branch, an allocation,
and a block.

Depending on the values of these parameters, the scoring function returns
different values for each function, and the ordering may be changed, which
makes the inliner take different decisions. The sensitivity of the scoring
function have a direct effect on the sorting of the candidates list, which
also have a strong impact on inline decisions.

To assure good decisions the system must have a good scoring function,
meaning that its result allows a good ordering for the inliner. But, a good
ordering depends on how accurate are the estimated scores. One way we can
calibrate the scores is to run the system on some known benchmarks. This way
the parameters can be calibrated to their best values, those who produce
the best estimates.

So there is a need to find a ``sweetspot'' on these values. The first idea is
to apply machine learning techniques to search for it through a space of
possible values. But there are also some issues, we cannot afford to have a
huge number of acquisition points because the whole system takes a long time
to process. Also, optimizing a function without knowing if it is differentiable,
or convex, is not a trivial machine learning task.
