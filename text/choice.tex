
The scoring function, as mentioned in ~\ref{inlining:scoring} estimates the
fitness of a function to be inlined, this way a function whose score is above
zero ($0$) may be inlined, if the budget allows. But the algorithm that decides
which function will be inlined is solely based in this information, not in the
current budget value, nor in the other scores computed.

In \cite{BerubePhD} the inlining algorithm sorts the candidates for inlining
based on the score function. As the score takes into account estimated values for
benefit and cost of inlining, we suggest to use it but aggregating the expenditure,
the budget, and other scores besides the most fitted.

The idea is to perform a knapsack algorithm considering all the scores above zero
and the budget (the ``size'' of the knapsack). Proceeding this way we can optimize
the amount of budget wasted for inlining.

